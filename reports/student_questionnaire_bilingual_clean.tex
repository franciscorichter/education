\documentclass[11pt]{article}
\usepackage[margin=1in]{geometry}
\usepackage[T1]{fontenc}
\usepackage[utf8]{inputenc}
\usepackage{microtype}
\usepackage{enumitem}
\usepackage{array}
\usepackage{booktabs}
\usepackage{longtable}
\usepackage{ragged2e}
\setlength{\parindent}{0pt}
\sloppy
\newcolumntype{L}{>{\RaggedRight\arraybackslash}p{0.48\textwidth}}
\begin{document}
\newpage
\begin{longtable}{@{}LL@{}}
\toprule
\textbf{Español} & \textbf{English} \\ 
\midrule
\endfirsthead
\toprule
\textbf{Español} & \textbf{English} \\ 
\midrule
\endhead
\bottomrule \\ 
\endfoot
\bottomrule
\multicolumn{2}{@{}l@{}}{\large\textbf{P01}} \\ 
¿Cuántos años tienes? & How old are you? \\
\textbf{Opciones}\par\begin{itemize}[leftmargin=*]\item 1: Menos de 10 años
\item 2: 10 años
\item 3: 11 años
\item 4: 12 años
\item 5: Más de 12 años\end{itemize} & \textbf{Options}\par\begin{itemize}[leftmargin=*]\item 1: Under 10 years
\item 2: 10 years
\item 3: 11 years
\item 4: 12 years
\item 5: Over 12 years\end{itemize} \\
\addlinespace[4pt]
\multicolumn{2}{@{}l@{}}{\large\textbf{P02}} \\ 
¿Cuál fue la primera lengua que aprendiste a hablar? & What was the first language you learned to speak? \\
\textbf{Opciones}\par\begin{itemize}[leftmargin=*]\item 1: Castellano
\item 2: Quechua
\item 3: Aimara
\item 4: Una lengua amazónica (awajún, shipibo, asháninka, etc.)
\item 5: Una lengua extranjera (inglés, francés, alemán, etc.)\end{itemize} & \textbf{Options}\par\begin{itemize}[leftmargin=*]\item 1: Spanish
\item 2: Quechua
\item 3: Aymara
\item 4: An Amazonian language (Awajún, Shipibo, Asháninka, etc.)
\item 5: A foreign language (English, French, German, etc.)\end{itemize} \\
\addlinespace[4pt]
\multicolumn{2}{@{}l@{}}{\large\textbf{P03}} \\ 
¿Qué lengua hablan en tu casa la mayor parte del tiempo? & What language is mostly spoken at home? \\
\textbf{Opciones}\par\begin{itemize}[leftmargin=*]\item 1: Castellano
\item 2: Quechua
\item 3: Aimara
\item 4: Una lengua amazónica (awajún, shipibo, asháninka, etc.)
\item 5: Una lengua extranjera (inglés, francés, alemán, etc.)\end{itemize} & \textbf{Options}\par\begin{itemize}[leftmargin=*]\item 1: Spanish
\item 2: Quechua
\item 3: Aymara
\item 4: An Amazonian language (Awajún, Shipibo, Asháninka, etc.)
\item 5: A foreign language (English, French, German, etc.)\end{itemize} \\
\addlinespace[4pt]
\multicolumn{2}{@{}l@{}}{\large\textbf{P04}} \\ 
Durante el último mes, ¿cuántos días has faltado a la escuela? & In the last month, how many days did you miss school? \\
\textbf{Opciones}\par\begin{itemize}[leftmargin=*]\item 1: De 0 a 1 día
\item 2: De 2 a 4 días
\item 3: De 5 a 7 días
\item 4: Más de 7 días\end{itemize} & \textbf{Options}\par\begin{itemize}[leftmargin=*]\item 1: From 0 to 1 day
\item 2: From 2 to 4 days
\item 3: From 5 to 7 days
\item 4: More than 7 days\end{itemize} \\
\addlinespace[4pt]
\multicolumn{2}{@{}l@{}}{\large\textbf{P05}} \\ 
¿Qué tan de acuerdo estás con las siguientes afirmaciones? & How much do you agree with the following statements? \\
\textbf{Sub-items}\par\begin{itemize}[leftmargin=*]\item p05\_01: Me emociona la idea de comenzar a estudiar la secundaria
\item p05\_02: Me da miedo comenzar la secundaria
\item p05\_03: Creo que mis notas serán mejores en la secundaria
\item p05\_04: Me gustaría que la escuela acabe en 6.° grado de primaria y no tener que estudiar más
\item p05\_05: Preferiría ponerme a trabajar en lugar de entrar a la secundaria
\item p05\_06: Siento que en la secundaria será muy difícil aprobar todos los cursos
\item p05\_07: Me preocupa no tener el dinero suficiente para seguir estudiando
\item p05\_08: Me preocupa sufrir de violencia o acoso escolar (bullying) en secundaria\end{itemize} & \textbf{Sub-items}\par\begin{itemize}[leftmargin=*]\item p05\_01: I am excited about starting secondary school
\item p05\_02: I am afraid to start secondary school
\item p05\_03: I think my grades will be better in secondary school
\item p05\_04: I would like school to end in 6th grade of primary and not have to study further
\item p05\_05: I would rather start working instead of entering secondary school
\item p05\_06: I feel it will be very hard to pass all courses in secondary school
\item p05\_07: I worry about not having enough money to keep studying
\item p05\_08: I worry about suffering violence or school bullying in secondary school\end{itemize} \\
\textbf{Opciones}\par\begin{itemize}[leftmargin=*]\item 1: Totalmente en desacuerdo
\item 2: En desacuerdo
\item 3: De acuerdo
\item 4: Totalmente de acuerdo\end{itemize} & \textbf{Options}\par\begin{itemize}[leftmargin=*]\item 1: Strongly disagree
\item 2: Disagree
\item 3: Agree
\item 4: Strongly agree\end{itemize} \\
\addlinespace[4pt]
\multicolumn{2}{@{}l@{}}{\large\textbf{P06}} \\ 
¿Qué tan preparado te sientes para comenzar el 1.° grado de secundaria el siguiente año? & How prepared do you feel to start 1st year of secondary school next year? \\
\textbf{Opciones}\par\begin{itemize}[leftmargin=*]\item 1: Nada preparado
\item 2: Poco preparado
\item 3: Más o menos preparado
\item 4: Totalmente preparado\end{itemize} & \textbf{Options}\par\begin{itemize}[leftmargin=*]\item 1: Not at all prepared
\item 2: A little prepared
\item 3: Somewhat prepared
\item 4: Fully prepared\end{itemize} \\
\addlinespace[4pt]
\multicolumn{2}{@{}l@{}}{\large\textbf{P07}} \\ 
¿Cuál crees que será el nivel de estudios más alto que alcanzarás? & What is the highest level of education you think you will achieve? \\
\textbf{Opciones}\par\begin{itemize}[leftmargin=*]\item 1: No terminaré la primaria
\item 2: Terminaré la secundaria
\item 3: Terminaré una carrera técnica
\item 4: Terminaré una carrera universitaria
\item 5: Terminaré una maestría o doctorado\end{itemize} & \textbf{Options}\par\begin{itemize}[leftmargin=*]\item 1: I will not finish primary school
\item 2: I will finish secondary school
\item 3: I will complete a technical degree
\item 4: I will complete a university degree
\item 5: I will complete a master's degree or doctorate\end{itemize} \\
\addlinespace[4pt]
\multicolumn{2}{@{}l@{}}{\large\textbf{P08}} \\ 
En una semana habitual, ¿cuántas horas les dedicas a la lectura en tu TIEMPO LIBRE? & In a typical week, how many hours do you spend reading in your FREE TIME? \\
\textbf{Opciones}\par\begin{itemize}[leftmargin=*]\item 1: No leo en mi tiempo libre
\item 2: Menos de 2 horas a la semana
\item 3: Entre 2 y 4 horas a la semana
\item 4: Entre 5 y 7 horas a la semana
\item 5: Entre 8 y 10 horas a la semana
\item 6: Más de 10 horas a la semana\end{itemize} & \textbf{Options}\par\begin{itemize}[leftmargin=*]\item 1: I do not read in my free time
\item 2: Less than 2 hours per week
\item 3: Between 2 and 4 hours per week
\item 4: Between 5 and 7 hours per week
\item 5: Between 8 and 10 hours per week
\item 6: More than 10 hours per week\end{itemize} \\
\addlinespace[4pt]
\multicolumn{2}{@{}l@{}}{\large\textbf{P09}} \\ 
¿Qué tan de acuerdo estás con los siguientes enunciados sobre las razones por las que lees en tu TIEMPO LIBRE? & How much do you agree with the following statements about why you read in your FREE TIME? \\
\textbf{Sub-items}\par\begin{itemize}[leftmargin=*]\item p09\_01: Leo porque lo disfruto mucho
\item p09\_02: Leo porque es divertido
\item p09\_03: Leo porque me gusta mucho
\item p09\_04: Leo porque creo que es interesante
\item p09\_05: Leo porque creo que la lectura es útil
\item p09\_06: Leo porque es importante para mí
\item p09\_07: Leo porque no quiero decepcionar a los demás
\item p09\_08: Leo porque me sentiré culpable si no lo hago
\item p09\_09: Leo porque eso es lo que los demás esperan que haga
\item p09\_10: Leo porque tengo que demostrarme a mí mismo que puedo sacar buenas notas en Comunicación
\item p09\_11: Leo porque me castigarán si no leo
\item p09\_12: Leo porque lo veo necesario para no sentirme mal conmigo mismo\end{itemize} & \textbf{Sub-items}\par\begin{itemize}[leftmargin=*]\item p09\_01: I read because I enjoy it a lot
\item p09\_02: I read because it is fun
\item p09\_03: I read because I like it a lot
\item p09\_04: I read because I think it is interesting
\item p09\_05: I read because I think reading is useful
\item p09\_06: I read because it is important to me
\item p09\_07: I read because I do not want to disappoint others
\item p09\_08: I read because I would feel guilty if I did not
\item p09\_09: I read because that is what others expect me to do
\item p09\_10: I read because I have to prove to myself that I can get good grades in Communication
\item p09\_11: I read because I will be punished if I do not
\item p09\_12: I read because I see it as necessary so I do not feel bad about myself\end{itemize} \\
\textbf{Opciones}\par\begin{itemize}[leftmargin=*]\item 1: Totalmente en desacuerdo
\item 2: En desacuerdo
\item 3: De acuerdo
\item 4: Totalmente de acuerdo\end{itemize} & \textbf{Options}\par\begin{itemize}[leftmargin=*]\item 1: Strongly disagree
\item 2: Disagree
\item 3: Agree
\item 4: Strongly agree\end{itemize} \\
\addlinespace[4pt]
\multicolumn{2}{@{}l@{}}{\large\textbf{P10}} \\ 
Cuando estudias un tema para aprender o para dar un examen, ¿qué tan seguido realizas las siguientes acciones? & When you study a topic to learn or for a test, how often do you do the following? \\
\textbf{Sub-items}\par\begin{itemize}[leftmargin=*]\item p10\_01: Resalto lo que me parece más importante
\item p10\_02: Intento memorizar el contenido palabra por palabra
\item p10\_03: Repito lo que he leído varias veces
\item p10\_04: Reescribo mis apuntes del libro a mi cuaderno
\item p10\_05: Intento comprender lo que aprendo
\item p10\_06: Se me ocurren ejemplos relacionados con lo que estoy aprendiendo
\item p10\_07: Intento relacionar lo que estoy aprendiendo con cosas que ya sé
\item p10\_08: Asocio lo que estudio con imágenes o situaciones que creo en mi mente
\item p10\_09: Intento explicar los conceptos con mis propias palabras
\item p10\_10: Intento deducir el significado de palabras o frases desconocidas utilizando el propio texto
\item p10\_11: Después de leer, me explico a mí mismo lo que he leído
\item p10\_12: Escribo resúmenes
\item p10\_13: Elaboro mapas conceptuales o diagramas
\item p10\_14: Explico de qué trata el texto a otros compañeros o personas\end{itemize} & \textbf{Sub-items}\par\begin{itemize}[leftmargin=*]\item p10\_01: I highlight what seems most important to me
\item p10\_02: I try to memorize the content word for word
\item p10\_03: I repeat what I have read several times
\item p10\_04: I rewrite my notes from the book into my notebook
\item p10\_05: I try to understand what I learn
\item p10\_06: I come up with examples related to what I am learning
\item p10\_07: I try to relate what I am learning to things I already know
\item p10\_08: I associate what I study with images or situations I create in my mind
\item p10\_09: I try to explain the concepts in my own words
\item p10\_10: I try to infer the meaning of unknown words or phrases using the text itself
\item p10\_11: After reading, I explain to myself what I have read
\item p10\_12: I write summaries
\item p10\_13: I make concept maps or diagrams
\item p10\_14: I explain what the text is about to classmates or others\end{itemize} \\
\textbf{Opciones}\par\begin{itemize}[leftmargin=*]\item 1: Nunca o casi nunca
\item 2: Algunas veces
\item 3: Muchas veces
\item 4: Siempre o casi siempre\end{itemize} & \textbf{Options}\par\begin{itemize}[leftmargin=*]\item 1: Never or almost never
\item 2: Sometimes
\item 3: Often
\item 4: Always or almost always\end{itemize} \\
\addlinespace[4pt]
\multicolumn{2}{@{}l@{}}{\large\textbf{P11}} \\ 
¿Qué tan de acuerdo estás con los siguientes enunciados sobre tu clase de COMUNICACIÓN? & How much do you agree with the following statements about your COMMUNICATION class? \\
\textbf{Sub-items}\par\begin{itemize}[leftmargin=*]\item p11\_01: Me gusta lo que leo en la clase
\item p11\_02: Los textos que nos dan en clase son muy difíciles de comprender
\item p11\_03: En la clase, me dan textos interesantes para leer
\item p11\_04: Disfruto aprender cosas nuevas en la clase
\item p11\_05: Los textos que nos dan en clase son muy largos
\item p11\_06: En general, me interesa lo que leo en la clase
\item p11\_07: Los textos que nos dan en clase tienen palabras muy difíciles
\item p11\_08: En las clases, me dan textos con información nueva
\item p11\_09: Los textos que leemos son para estudiantes de nuestra edad
\item p11\_10: Los textos que nos dan en clase son muy aburridos\end{itemize} & \textbf{Sub-items}\par\begin{itemize}[leftmargin=*]\item p11\_01: I like what I read in class
\item p11\_02: The texts we are given in class are very difficult to understand
\item p11\_03: In class, we are given interesting texts to read
\item p11\_04: I enjoy learning new things in class
\item p11\_05: The texts we are given in class are very long
\item p11\_06: In general, I am interested in what I read in class
\item p11\_07: The texts we are given in class have very difficult words
\item p11\_08: In class, we are given texts with new information
\item p11\_09: The texts we read are for students our age
\item p11\_10: The texts we are given in class are very boring\end{itemize} \\
\textbf{Opciones}\par\begin{itemize}[leftmargin=*]\item 1: Totalmente en desacuerdo
\item 2: En desacuerdo
\item 3: De acuerdo
\item 4: Totalmente de acuerdo\end{itemize} & \textbf{Options}\par\begin{itemize}[leftmargin=*]\item 1: Strongly disagree
\item 2: Disagree
\item 3: Agree
\item 4: Strongly agree\end{itemize} \\
\addlinespace[4pt]
\multicolumn{2}{@{}l@{}}{\large\textbf{P12}} \\ 
¿Qué tan seguido el profesor que enseña COMUNICACIÓN realiza las siguientes acciones? & How often does the teacher who teaches COMMUNICATION do the following? \\
\textbf{Sub-items}\par\begin{itemize}[leftmargin=*]\item p12\_01: Nos pide dar nuestra opinión personal sobre lo que leemos
\item p12\_02: Nos pide explicar por qué nos gusta o disgusta un pasaje del texto
\item p12\_03: Nos pide que pensemos o que nos preguntemos sobre cómo se relacionan los textos que leemos con nuestra vida diaria
\item p12\_04: Nos pide que expliquemos nuestras respuestas con detalle
\item p12\_05: Nos presenta textos que incluyen esquemas o mapas conceptuales
\item p12\_06: Nos pide explicar el aporte de los gráficos o dibujos del texto
\item p12\_07: Nos presenta diferentes textos que tratan del mismo tema
\item p12\_08: Nos hace pensar si es que podemos confiar en lo que dice el autor en el texto
\item p12\_09: Nos pide que compartamos en grupo nuestras opiniones sobre los textos\end{itemize} & \textbf{Sub-items}\par\begin{itemize}[leftmargin=*]\item p12\_01: He/She asks us to give our personal opinion about what we read
\item p12\_02: He/She asks us to explain why we like or dislike a passage of the text
\item p12\_03: He/She asks us to think about how the texts we read relate to our daily life
\item p12\_04: He/She asks us to explain our answers in detail
\item p12\_05: He/She presents texts that include outlines or concept maps
\item p12\_06: He/She asks us to explain the contribution of the text's graphics or drawings
\item p12\_07: He/She presents different texts on the same topic
\item p12\_08: He/She makes us consider whether we can trust what the author says in the text
\item p12\_09: He/She asks us to share our opinions about the texts in groups\end{itemize} \\
\textbf{Opciones}\par\begin{itemize}[leftmargin=*]\item 1: Nunca o casi nunca
\item 2: En algunas clases
\item 3: En la mayoría de las clases
\item 4: En todas o en casi todas las clases\end{itemize} & \textbf{Options}\par\begin{itemize}[leftmargin=*]\item 1: Never or almost never
\item 2: In some classes
\item 3: In most classes
\item 4: In all or almost all classes\end{itemize} \\
\addlinespace[4pt]
\multicolumn{2}{@{}l@{}}{\large\textbf{P13}} \\ 
¿Qué tan seguido el profesor que enseña MATEMÁTICA realiza las siguientes acciones? & How often does the teacher who teaches MATHEMATICS do the following? \\
\textbf{Sub-items}\par\begin{itemize}[leftmargin=*]\item p13\_01: Nos deja tareas que requieren pensar más y no solo aplicar fórmulas o resolver ecuaciones
\item p13\_02: Nos pide explicar cómo hemos resuelto un problema o ejercicio
\item p13\_03: Nos pide que pensemos o nos preguntemos sobre cómo se relacionan los temas nuevos con los anteriores
\item p13\_04: Nos pide que expliquemos nuestras respuestas con detalle
\item p13\_05: Nos enseña a resolver los problemas de muchas maneras
\item p13\_06: Nos pide que elaboremos nuestros propios problemas de matemática
\item p13\_07: Nos motiva a que encontremos nuestra propia forma de resolver los problemas o ejercicios de matemática
\item p13\_08: Nos motiva a relacionar lo que aprendemos con situaciones de la vida real\end{itemize} & \textbf{Sub-items}\par\begin{itemize}[leftmargin=*]\item p13\_01: He/She assigns tasks that require more thinking and not just applying formulas or solving equations
\item p13\_02: He/She asks us to explain how we solved a problem or exercise
\item p13\_03: He/She asks us to think about how new topics relate to previous ones
\item p13\_04: He/She asks us to explain our answers in detail
\item p13\_05: He/She teaches us to solve problems in many ways
\item p13\_06: He/She asks us to create our own math problems
\item p13\_07: He/She encourages us to find our own way to solve math problems or exercises
\item p13\_08: He/She encourages us to relate what we learn to real-life situations\end{itemize} \\
\textbf{Opciones}\par\begin{itemize}[leftmargin=*]\item 1: Nunca o casi nunca
\item 2: En algunas clases
\item 3: En la mayoría de las clases
\item 4: En todas o en casi todas las clases\end{itemize} & \textbf{Options}\par\begin{itemize}[leftmargin=*]\item 1: Never or almost never
\item 2: In some classes
\item 3: In most classes
\item 4: In all or almost all classes\end{itemize} \\
\addlinespace[4pt]
\multicolumn{2}{@{}l@{}}{\large\textbf{P14}} \\ 
¿Qué tan seguido ocurren las siguientes situaciones en las clases de MATEMÁTICA? & How often do the following situations occur in MATH classes? \\
\textbf{Sub-items}\par\begin{itemize}[leftmargin=*]\item p14\_01: Durante la clase hay desorden en el salón
\item p14\_02: El profesor tiene que esperar mucho tiempo para que los estudiantes se callen
\item p14\_03: Algunos estudiantes hacen tanto desorden que se hace difícil aprender
\item p14\_04: Cuando el profesor explica algo en clase, es capaz de hacer que todos le prestemos atención
\item p14\_05: El profesor mantiene el orden de las clases a pesar de todas las preguntas de los estudiantes
\item p14\_06: Los estudiantes participan de manera ordenada en las clases
\item p14\_07: Cuando un estudiante participa en clase, otros interrumpen
\item p14\_08: Los estudiantes se distraen en la clase\end{itemize} & \textbf{Sub-items}\par\begin{itemize}[leftmargin=*]\item p14\_01: During class there is disorder in the classroom
\item p14\_02: The teacher has to wait a long time for students to be quiet
\item p14\_03: Some students create so much disorder that it is hard to learn
\item p14\_04: When the teacher explains something in class, he/she can get everyone to pay attention
\item p14\_05: The teacher keeps order in class despite all the students' questions
\item p14\_06: Students participate in an orderly way in class
\item p14\_07: When a student participates in class, others interrupt
\item p14\_08: Students get distracted in class\end{itemize} \\
\textbf{Opciones}\par\begin{itemize}[leftmargin=*]\item 1: Nunca o casi nunca
\item 2: En algunas clases
\item 3: En la mayoría de las clases
\item 4: En todas o en casi todas las clases\end{itemize} & \textbf{Options}\par\begin{itemize}[leftmargin=*]\item 1: Never or almost never
\item 2: In some classes
\item 3: In most classes
\item 4: In all or almost all classes\end{itemize} \\
\addlinespace[4pt]
\multicolumn{2}{@{}l@{}}{\large\textbf{P15}} \\ 
¿Qué tan de acuerdo estás con los siguientes enunciados? & How much do you agree with the following statements? \\
\textbf{Sub-items}\par\begin{itemize}[leftmargin=*]\item p15\_01: La matemática me da miedo
\item p15\_02: En general, me preocupa resolver ejercicios de matemática
\item p15\_03: Me pongo muy nervioso o nerviosa en un examen de matemática
\item p15\_04: Normalmente, la matemática me pone muy nervioso o nerviosa
\item p15\_05: Me siento mal cuando pienso en resolver problemas de matemática
\item p15\_06: Cuando hago problemas de matemática se me pone la mente en blanco y no puedo pensar claramente
\item p15\_07: Me dan miedo los exámenes de matemática
\item p15\_08: Las tareas de matemática me hacen sentir nervioso o nerviosa\end{itemize} & \textbf{Sub-items}\par\begin{itemize}[leftmargin=*]\item p15\_01: Math scares me
\item p15\_02: In general, I worry about solving math exercises
\item p15\_03: I get very nervous in a math exam
\item p15\_04: Normally, math makes me very nervous
\item p15\_05: I feel bad when I think about solving math problems
\item p15\_06: When I do math problems my mind goes blank and I can't think clearly
\item p15\_07: I am afraid of math exams
\item p15\_08: Math homework makes me feel nervous\end{itemize} \\
\textbf{Opciones}\par\begin{itemize}[leftmargin=*]\item 1: Totalmente en desacuerdo
\item 2: En desacuerdo
\item 3: De acuerdo
\item 4: Totalmente de acuerdo\end{itemize} & \textbf{Options}\par\begin{itemize}[leftmargin=*]\item 1: Strongly disagree
\item 2: Disagree
\item 3: Agree
\item 4: Strongly agree\end{itemize} \\
\addlinespace[4pt]
\multicolumn{2}{@{}l@{}}{\large\textbf{P16}} \\ 
¿Qué tan de acuerdo estás con las siguientes afirmaciones relacionadas con la matemática? & How much do you agree with the following statements related to mathematics? \\
\textbf{Sub-items}\par\begin{itemize}[leftmargin=*]\item p16\_01: Me gusta la matemática
\item p16\_02: Me siento bien cuando tengo que hacer mis tareas de matemática
\item p16\_03: Me da pena tener que gastar mi tiempo haciendo ejercicios de matemática
\item p16\_04: Cuando tengo que hacer matemática, preferiría hacer otras cosas
\item p16\_05: La matemática es fácil para mí
\item p16\_06: Me intereso por la clase de matemática
\item p16\_07: Me interesa que me vaya bien en matemática porque es importante para mí
\item p16\_08: Si me dieran la opción de elegir, escogería aprender matemática\end{itemize} & \textbf{Sub-items}\par\begin{itemize}[leftmargin=*]\item p16\_01: I like math
\item p16\_02: I feel good when I have to do my math homework
\item p16\_03: I feel sorry to spend my time doing math exercises
\item p16\_04: When I have to do math, I would rather do other things
\item p16\_05: Math is easy for me
\item p16\_06: I am interested in math class
\item p16\_07: It matters to me to do well in math because it is important to me
\item p16\_08: If I had the option, I would choose to learn math\end{itemize} \\
\textbf{Opciones}\par\begin{itemize}[leftmargin=*]\item 1: Totalmente en desacuerdo
\item 2: En desacuerdo
\item 3: De acuerdo
\item 4: Totalmente de acuerdo\end{itemize} & \textbf{Options}\par\begin{itemize}[leftmargin=*]\item 1: Strongly disagree
\item 2: Disagree
\item 3: Agree
\item 4: Strongly agree\end{itemize} \\
\addlinespace[4pt]
\multicolumn{2}{@{}l@{}}{\large\textbf{P17}} \\ 
¿Qué tan seguido el profesor o profesora que enseña PERSONAL SOCIAL hace lo siguiente? & How often does the SOCIAL PERSONAL teacher do the following? \\
\textbf{Sub-items}\par\begin{itemize}[leftmargin=*]\item p17\_01: Nos pide recordar de memoria las fechas y los lugares en los que sucedieron los hechos
\item p17\_02: Nos pide identificar las diversas causas y consecuencias de un hecho
\item p17\_03: Nos pide aprender de memoria los hechos tal como los explicó
\item p17\_04: Nos presenta diversas maneras de explicar un mismo hecho
\item p17\_05: Nos dice que solo existe una verdadera versión de los hechos
\item p17\_06: Nos pide utilizar información confiable al defender un punto de vista
\item p17\_07: Nos dice que los personajes importantes son los que solucionan los problemas de la sociedad\end{itemize} & \textbf{Sub-items}\par\begin{itemize}[leftmargin=*]\item p17\_01: He/She asks us to memorize the dates and places where events happened
\item p17\_02: He/She asks us to identify the various causes and consequences of an event
\item p17\_03: He/She asks us to memorize the facts as explained
\item p17\_04: He/She presents different ways of explaining the same event
\item p17\_05: He/She says there is only one true version of events
\item p17\_06: He/She asks us to use reliable information when defending a point of view
\item p17\_07: He/She says important figures are the ones who solve society's problems\end{itemize} \\
\textbf{Opciones}\par\begin{itemize}[leftmargin=*]\item 1: Nunca o casi nunca
\item 2: En algunas clases
\item 3: En la mayoría de las clases
\item 4: En todas o en casi todas las clases\end{itemize} & \textbf{Options}\par\begin{itemize}[leftmargin=*]\item 1: Never or almost never
\item 2: In some classes
\item 3: In most classes
\item 4: In all or almost all classes\end{itemize} \\
\addlinespace[4pt]
\multicolumn{2}{@{}l@{}}{\large\textbf{P18}} \\ 
¿Qué tan de acuerdo estás con las siguientes afirmaciones? & How much do you agree with the following statements? \\
\textbf{Sub-items}\par\begin{itemize}[leftmargin=*]\item p18\_01: Considero que todas las personas de la comunidad influyen sobre los hechos que ocurren en la sociedad
\item p18\_02: Considero que las decisiones de los personajes importantes son las que determinan lo que ocurre en la sociedad
\item p18\_03: Para comprender un hecho es necesario conocer las características de la época y del lugar en el que ocurre
\item p18\_04: Algunas ideas y costumbres del pasado se mantienen hasta la actualidad
\item p18\_05: Las ideas y costumbres de la actualidad son totalmente distintas a las del pasado
\item p18\_06: Considero que un hecho puede tener varias causas
\item p18\_07: Considero que un hecho puede tener varias consecuencias\end{itemize} & \textbf{Sub-items}\par\begin{itemize}[leftmargin=*]\item p18\_01: I believe everyone in the community influences the events that occur in society
\item p18\_02: I believe the decisions of important figures determine what happens in society
\item p18\_03: To understand an event it is necessary to know the characteristics of the time and place in which it occurs
\item p18\_04: Some ideas and customs from the past remain to this day
\item p18\_05: Today's ideas and customs are completely different from those of the past
\item p18\_06: I believe an event can have several causes
\item p18\_07: I believe an event can have several consequences\end{itemize} \\
\textbf{Opciones}\par\begin{itemize}[leftmargin=*]\item 1: Totalmente en desacuerdo
\item 2: En desacuerdo
\item 3: De acuerdo
\item 4: Totalmente de acuerdo\end{itemize} & \textbf{Options}\par\begin{itemize}[leftmargin=*]\item 1: Strongly disagree
\item 2: Disagree
\item 3: Agree
\item 4: Strongly agree\end{itemize} \\
\addlinespace[4pt]
\multicolumn{2}{@{}l@{}}{\large\textbf{P19}} \\ 
¿Qué tan de acuerdo estás con las siguientes afirmaciones? & How much do you agree with the following statements? \\
\textbf{Sub-items}\par\begin{itemize}[leftmargin=*]\item p19\_01: Considero que para comprender un hecho es necesario utilizar información de diversas fuentes
\item p19\_02: Considero que cualquier información es confiable, sin importar de donde proviene
\item p19\_03: Las personas explican los hechos desde su punto de vista
\item p19\_04: Un mismo hecho de la sociedad puede ser comprendido de distinta manera
\item p19\_05: Existe una sola versión verdadera de los hechos que ocurren en la sociedad
\item p19\_06: Para defender un punto de vista sobre algún hecho es necesario usar información confiable
\item p19\_07: Los problemas sociales se pueden resolver de una sola manera
\item p19\_08: Para solucionar los problemas de la sociedad se debe hacer lo mismo que los países desarrollados
\item p19\_09: Todas las personas pueden contribuir a la solución de los problemas
\item p19\_10: La solución a los problemas de la sociedad depende de los personajes importantes\end{itemize} & \textbf{Sub-items}\par\begin{itemize}[leftmargin=*]\item p19\_01: I believe that to understand an event it is necessary to use information from various sources
\item p19\_02: I believe that any information is reliable regardless of where it comes from
\item p19\_03: People explain events from their point of view
\item p19\_04: The same event in society can be understood in different ways
\item p19\_05: There is only one true version of the events that occur in society
\item p19\_06: To defend a point of view about an event it is necessary to use reliable information
\item p19\_07: Social problems can be solved in only one way
\item p19\_08: To solve society's problems we must do the same as developed countries
\item p19\_09: Everyone can contribute to solving problems
\item p19\_10: The solution to society's problems depends on important figures\end{itemize} \\
\textbf{Opciones}\par\begin{itemize}[leftmargin=*]\item 1: Totalmente en desacuerdo
\item 2: En desacuerdo
\item 3: De acuerdo
\item 4: Totalmente de acuerdo\end{itemize} & \textbf{Options}\par\begin{itemize}[leftmargin=*]\item 1: Strongly disagree
\item 2: Disagree
\item 3: Agree
\item 4: Strongly agree\end{itemize} \\
\addlinespace[4pt]
\multicolumn{2}{@{}l@{}}{\large\textbf{P20}} \\ 
Acerca de tu escuela, ¿qué tan de acuerdo estás con las siguientes afirmaciones? & About your school, how much do you agree with the following statements? \\
\textbf{Sub-items}\par\begin{itemize}[leftmargin=*]\item p20\_01: Prefiero faltar a la escuela
\item p20\_02: Mi escuela es un lugar donde me siento solo
\item p20\_03: Preferiría estudiar en otra escuela
\item p20\_04: Siento que no pertenezco a esta escuela
\item p20\_05: Me siento orgulloso de estar en esta escuela
\item p20\_06: Mi escuela es un lugar donde me siento como un extraño
\item p20\_07: Me siento seguro en el camino cuando voy de mi casa a la escuela
\item p20\_08: Me siento seguro en los salones de clase de mi escuela
\item p20\_09: Me siento seguro en otros lugares de la escuela (por ejemplo, patios, pasillos, baños, etc.)\end{itemize} & \textbf{Sub-items}\par\begin{itemize}[leftmargin=*]\item p20\_01: I prefer to skip school
\item p20\_02: My school is a place where I feel lonely
\item p20\_03: I would prefer to study at another school
\item p20\_04: I feel that I do not belong at this school
\item p20\_05: I feel proud to be at this school
\item p20\_06: My school is a place where I feel like a stranger
\item p20\_07: I feel safe on the way when I go from my home to school
\item p20\_08: I feel safe in my school's classrooms
\item p20\_09: I feel safe in other places at school (e.g., yards, hallways, bathrooms, etc.)\end{itemize} \\
\textbf{Opciones}\par\begin{itemize}[leftmargin=*]\item 1: Totalmente en desacuerdo
\item 2: En desacuerdo
\item 3: De acuerdo
\item 4: Totalmente de acuerdo\end{itemize} & \textbf{Options}\par\begin{itemize}[leftmargin=*]\item 1: Strongly disagree
\item 2: Disagree
\item 3: Agree
\item 4: Strongly agree\end{itemize} \\
\addlinespace[4pt]
\multicolumn{2}{@{}l@{}}{\large\textbf{P21}} \\ 
¿Qué tan de acuerdo estas con las siguientes afirmaciones acerca de tus profesores? & How much do you agree with the following statements about your teachers? \\
\textbf{Sub-items}\par\begin{itemize}[leftmargin=*]\item p21\_01: Los profesores echan la culpa a los estudiantes por las cosas que pasan
\item p21\_02: Los profesores pueden castigar a los estudiantes sin decirles la razón por la que se les está castigando
\item p21\_03: Los profesores tienen demasiadas normas que no tienen sentido, pero debo obedecerlas
\item p21\_04: En general, los profesores de esta escuela no son muy pacientes con los alumnos
\item p21\_05: Los profesores dejan que los estudiantes les respondan de forma inadecuada o insulten\end{itemize} & \textbf{Sub-items}\par\begin{itemize}[leftmargin=*]\item p21\_01: Teachers blame students for things that happen
\item p21\_02: Teachers can punish students without telling them the reason they are being punished
\item p21\_03: Teachers have too many rules that do not make sense, but I must obey them
\item p21\_04: In general, teachers at this school are not very patient with students
\item p21\_05: Teachers allow students to respond to them inappropriately or insult them\end{itemize} \\
\textbf{Opciones}\par\begin{itemize}[leftmargin=*]\item 1: Totalmente en desacuerdo
\item 2: En desacuerdo
\item 3: De acuerdo
\item 4: Totalmente de acuerdo\end{itemize} & \textbf{Options}\par\begin{itemize}[leftmargin=*]\item 1: Strongly disagree
\item 2: Disagree
\item 3: Agree
\item 4: Strongly agree\end{itemize} \\
\addlinespace[4pt]
\multicolumn{2}{@{}l@{}}{\large\textbf{P22}} \\ 
¿Qué tan de acuerdo estás con las siguientes afirmaciones sobre tu participación en las clases? & How much do you agree with the following statements about your participation in class? \\
\textbf{Sub-items}\par\begin{itemize}[leftmargin=*]\item p22\_01: Me parece importante participar en las clases para aprender
\item p22\_02: Prefiero no participar en las clases para evitar errores y burlas de mis compañeros
\item p22\_03: Pregunto en las clases cuando no comprendo algo
\item p22\_04: Expreso mis opiniones en las clases aun cuando estas sean diferentes a las demás\end{itemize} & \textbf{Sub-items}\par\begin{itemize}[leftmargin=*]\item p22\_01: I think it is important to participate in class to learn
\item p22\_02: I prefer not to participate in class to avoid mistakes and classmates' teasing
\item p22\_03: I ask questions in class when I do not understand something
\item p22\_04: I express my opinions in class even when they are different from others'\end{itemize} \\
\textbf{Opciones}\par\begin{itemize}[leftmargin=*]\item 1: Totalmente en desacuerdo
\item 2: En desacuerdo
\item 3: De acuerdo
\item 4: Totalmente de acuerdo\end{itemize} & \textbf{Options}\par\begin{itemize}[leftmargin=*]\item 1: Strongly disagree
\item 2: Disagree
\item 3: Agree
\item 4: Strongly agree\end{itemize} \\
\addlinespace[4pt]
\multicolumn{2}{@{}l@{}}{\large\textbf{P23}} \\ 
Durante las clases, ¿qué tan seguido tus profesores hacen lo siguiente? & During classes, how often do your teachers do the following? \\
\textbf{Sub-items}\par\begin{itemize}[leftmargin=*]\item p23\_01: Nos motivan a expresar nuestras ideas y opiniones en clase
\item p23\_02: Se molestan cuando nos equivocamos al participar en clase
\item p23\_03: Nos hablan mucho y dejan poco espacio para que participemos
\item p23\_04: Nos escuchan con interés cuando opinamos en clase
\item p23\_05: Nos motivan a formular preguntas durante la clase
\item p23\_06: Nos plantean actividades interesantes y entretenidas que motivan nuestra participación
\item p23\_07: Nos piden que opinemos al final de la clase para no interrumpirlos
\item p23\_08: Nos demuestran que prefieren la participación de los estudiantes más destacados\end{itemize} & \textbf{Sub-items}\par\begin{itemize}[leftmargin=*]\item p23\_01: They encourage us to express our ideas and opinions in class
\item p23\_02: They get upset when we make mistakes while participating in class
\item p23\_03: They talk a lot and leave little room for us to participate
\item p23\_04: They listen with interest when we give our opinions in class
\item p23\_05: They encourage us to ask questions during class
\item p23\_06: They propose interesting and engaging activities that motivate our participation
\item p23\_07: They ask us to share our opinions at the end of class so as not to interrupt
\item p23\_08: They show that they prefer the participation of the top students\end{itemize} \\
\textbf{Opciones}\par\begin{itemize}[leftmargin=*]\item 1: Nunca o casi nunca
\item 2: En algunas clases
\item 3: En la mayoría de las clases
\item 4: En todas o en casi todas las clases\end{itemize} & \textbf{Options}\par\begin{itemize}[leftmargin=*]\item 1: Never or almost never
\item 2: In some classes
\item 3: In most classes
\item 4: In all or almost all classes\end{itemize} \\
\addlinespace[4pt]
\multicolumn{2}{@{}l@{}}{\large\textbf{P24}} \\ 
Con respecto a tu experiencia escolar este año, ¿con qué frecuencia te sientes de la siguiente manera? & Regarding your school experience this year, how often do you feel the following way? \\
\textbf{Sub-items}\par\begin{itemize}[leftmargin=*]\item p24\_01: Siento que hay demasiadas tareas para la casa
\item p24\_02: Siento que decepciono a mis padres cuando mis notas en los exámenes son malas
\item p24\_03: Siento que decepciono a mi profesor cuando mis notas en los exámenes son bajas
\item p24\_04: Siento que hay demasiadas tareas o actividades que hacer en la escuela
\item p24\_05: Me preocupa que mis notas sean malas
\item p24\_06: Siento que hay demasiados exámenes en la escuela
\item p24\_07: Pienso que la nota es muy importante para mi futuro e incluso puede determinar toda mi vida
\item p24\_08: Siento que en mi colegio me dejan mucha tarea para la casa\end{itemize} & \textbf{Sub-items}\par\begin{itemize}[leftmargin=*]\item p24\_01: I feel there is too much homework
\item p24\_02: I feel I disappoint my parents when my test scores are bad
\item p24\_03: I feel I disappoint my teacher when my test scores are low
\item p24\_04: I feel there is too much homework or too many activities to do at school
\item p24\_05: I worry that my grades are bad
\item p24\_06: I feel there are too many tests at school
\item p24\_07: I think grades are very important for my future and can even determine my whole life
\item p24\_08: I feel my school gives me a lot of homework\end{itemize} \\
\textbf{Opciones}\par\begin{itemize}[leftmargin=*]\item 1: Nunca o casi nunca
\item 2: Algunas veces
\item 3: Muchas veces
\item 4: Siempre o casi siempre\end{itemize} & \textbf{Options}\par\begin{itemize}[leftmargin=*]\item 1: Never or almost never
\item 2: Sometimes
\item 3: Often
\item 4: Always or almost always\end{itemize} \\
\addlinespace[4pt]
\multicolumn{2}{@{}l@{}}{\large\textbf{P25}} \\ 
Durante este año, ¿con qué frecuencia un PROFESOR(A) U OTRO ADULTO de tu colegio ha actuado de la siguiente manera? Piensa en situaciones que incluyan contacto cara a cara, mensajes de texto, WhatsApp, redes sociales o por otro medio por internet. & During this year, how often has a TEACHER OR ANOTHER ADULT at your school acted in the following way? Think of situations including face-to-face contact, text messages, WhatsApp, social media, or other internet means. \\
\textbf{Sub-items}\par\begin{itemize}[leftmargin=*]\item p25\_01: Ha amenazado o intimidado a un estudiante
\item p25\_02: Se ha burlado o ha insultado a un estudiante
\item p25\_03: Ha castigado a un estudiante golpeándole con su mano u otra parte de su cuerpo (le ha dado un cocacho, un palmazo, una cachetada, una patada, etc.)
\item p25\_04: Ha castigado a un estudiante golpeándole con un objeto (cuaderno, correa, regla, palo, etc.)
\item p25\_05: Ha castigado a un estudiante humillándolo o haciéndolo sentir mal (gritándole, ridiculizándolo delante de sus compañeros)\end{itemize} & \textbf{Sub-items}\par\begin{itemize}[leftmargin=*]\item p25\_01: Has threatened or intimidated a student
\item p25\_02: Has made fun of or insulted a student
\item p25\_03: Has punished a student by hitting him/her with his/her hand or another part of the body (gave a knock, a slap, a smack, a kick, etc.)
\item p25\_04: Has punished a student by hitting him/her with an object (notebook, belt, ruler, stick, etc.)
\item p25\_05: Has punished a student by humiliating him/her or making him/her feel bad (yelling, ridiculing him/her in front of classmates)\end{itemize} \\
\textbf{Opciones}\par\begin{itemize}[leftmargin=*]\item 1: Nunca
\item 2: Unas cuantas veces al año
\item 3: Unas cuantas veces al mes
\item 4: Una o más veces por semana\end{itemize} & \textbf{Options}\par\begin{itemize}[leftmargin=*]\item 1: Never
\item 2: A few times a year
\item 3: A few times a month
\item 4: Once or more times per week\end{itemize} \\
\addlinespace[4pt]
\multicolumn{2}{@{}l@{}}{\large\textbf{P26}} \\ 
Durante este año, ¿con qué frecuencia ALGUNOS COMPAÑEROS(AS) han actuado de la siguiente manera con otros(as) estudiantes? Piensa en situaciones que incluyan contacto cara a cara, mensajes de texto, WhatsApp, redes sociales o por otro medio por internet. & During this year, how often have SOME CLASSMATES acted in the following way toward other students? Think of situations including face-to-face contact, text messages, WhatsApp, social media, or other internet means. \\
\textbf{Sub-items}\par\begin{itemize}[leftmargin=*]\item p26\_01: Han amenazado o intimidado a un estudiante
\item p26\_02: Han golpeado o herido a un estudiante (jalándole el pelo, dándole cocachos, pateándolo, haciéndolo tropezar, empujándolo, etc.)
\item p26\_03: Se han burlado o insultado a un estudiante
\item p26\_04: Han quitado, escondido, robado o roto sus cosas a un estudiante
\item p26\_05: Han difundido rumores o chismes sobre un estudiante
\item p26\_06: Han gritado a un estudiante frente a otros\end{itemize} & \textbf{Sub-items}\par\begin{itemize}[leftmargin=*]\item p26\_01: Have threatened or intimidated a student
\item p26\_02: Have hit or injured a student (pulling their hair, giving knocks, kicking, tripping, pushing, etc.)
\item p26\_03: Have made fun of or insulted a student
\item p26\_04: Have taken, hidden, stolen or broken a student's belongings
\item p26\_05: Have spread rumors or gossip about a student
\item p26\_06: Have shouted at a student in front of others\end{itemize} \\
\textbf{Opciones}\par\begin{itemize}[leftmargin=*]\item 1: Nunca
\item 2: Unas cuantas veces al año
\item 3: Unas cuantas veces al mes
\item 4: Una o más veces por semana\end{itemize} & \textbf{Options}\par\begin{itemize}[leftmargin=*]\item 1: Never
\item 2: A few times a year
\item 3: A few times a month
\item 4: Once or more times per week\end{itemize} \\
\addlinespace[4pt]
\multicolumn{2}{@{}l@{}}{\large\textbf{P27}} \\ 
Cuando las situaciones mencionadas en las preguntas anteriores han sucedido en tu colegio, ¿le contaste y/o solicitaste ayuda a las siguientes personas? & When the situations mentioned in the previous questions happened at your school, did you tell and/or ask the following people for help? \\
\textbf{Sub-items}\par\begin{itemize}[leftmargin=*]\item p27\_01: Mi tutor(a)
\item p27\_02: Otro profesor(a)
\item p27\_03: Al director(a)
\item p27\_04: Al auxiliar
\item p27\_05: Al psicólogo(a) de la escuela
\item p27\_06: Al personal administrativo, de limpieza o seguridad
\item p27\_07: A un familiar
\item p27\_08: A un amigo(a)\end{itemize} & \textbf{Sub-items}\par\begin{itemize}[leftmargin=*]\item p27\_01: My homeroom teacher
\item p27\_02: Another teacher
\item p27\_03: The principal
\item p27\_04: The assistant
\item p27\_05: The school psychologist
\item p27\_06: Administrative, cleaning or security staff
\item p27\_07: A family member
\item p27\_08: A friend\end{itemize} \\
\textbf{Opciones}\par\begin{itemize}[leftmargin=*]\item 1: No
\item 2: Sí\end{itemize} & \textbf{Options}\par\begin{itemize}[leftmargin=*]\item 1: No
\item 2: Yes\end{itemize} \\
\addlinespace[4pt]
\end{longtable}
\end{document}
