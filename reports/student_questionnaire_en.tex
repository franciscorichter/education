\documentclass[11pt]{article}
\usepackage[margin=1in]{geometry}
\usepackage{enumitem}
\usepackage[T1]{fontenc}
\usepackage[utf8]{inputenc}
\title{ENLA 2024 — Student Questionnaire (English Translation)}
\date{Generated on 2025-10-12 01:42}
\begin{document}
\maketitle
\tableofcontents
\newpage
\section*{P01: ¿Cuántos años tienes?}
\subsection*{Options}
\begin{itemize}[leftmargin=*]
  \item 1: Menos de 10 años
  \item 2: 10 años
  \item 3: 11 años
  \item 4: 12 años
  \item 5: Más de 12 años
\end{itemize}
\bigskip
\section*{P02: ¿Cuál fue la primera lengua que aprendiste a hablar?}
\subsection*{Options}
\begin{itemize}[leftmargin=*]
  \item 1: Castellano
  \item 2: Quechua
  \item 3: Aimara
  \item 4: Una lengua amazónica (awajún, shipibo, asháninka, etc.)
  \item 5: Una lengua extranjera (inglés, francés, alemán, etc.)
\end{itemize}
\bigskip
\section*{P03: ¿Qué lengua hablan en tu casa la mayor parte del tiempo?}
\subsection*{Options}
\begin{itemize}[leftmargin=*]
  \item 1: Castellano
  \item 2: Quechua
  \item 3: Aimara
  \item 4: Una lengua amazónica (awajún, shipibo, asháninka, etc.)
  \item 5: Una lengua extranjera (inglés, francés, alemán, etc.)
\end{itemize}
\bigskip
\section*{P04: Durante el último mes, ¿cuántos días has faltado a la escuela?}
\subsection*{Options}
\begin{itemize}[leftmargin=*]
  \item 1: De 0 a 1 día
  \item 2: De 2 a 4 días
  \item 3: De 5 a 7 días
  \item 4: Más de 7 días
\end{itemize}
\bigskip
\section*{P05: ¿Qué tan de acuerdo estás con las siguientes afirmaciones?}
\subsection*{Sub-items}
\begin{itemize}[leftmargin=*]
  \item p05\_01: Me emociona la idea de comenzar a estudiar la secundaria
  \item p05\_02: Me da miedo comenzar la secundaria
  \item p05\_03: Creo que mis notas serán mejores en la secundaria
  \item p05\_04: Me gustaría que la escuela acabe en 6.° grado de primaria y no tener que estudiar más
  \item p05\_05: Preferiría ponerme a trabajar en lugar de entrar a la secundaria
  \item p05\_06: Siento que en la secundaria será muy difícil aprobar todos los cursos
  \item p05\_07: Me preocupa no tener el dinero suficiente para seguir estudiando
  \item p05\_08: Me preocupa sufrir de violencia o acoso escolar (bullying) en secundaria
\end{itemize}
\subsection*{Options}
\begin{itemize}[leftmargin=*]
  \item 1: Totalmente en desacuerdo
  \item 2: En desacuerdo
  \item 3: De acuerdo
  \item 4: Totalmente de acuerdo
\end{itemize}
\bigskip
\section*{P06: ¿Qué tan preparado te sientes para comenzar el 1.° grado de secundaria el siguiente año?}
\subsection*{Options}
\begin{itemize}[leftmargin=*]
  \item 1: Nada preparado
  \item 2: Poco preparado
  \item 3: Más o menos preparado
  \item 4: Totalmente preparado
\end{itemize}
\bigskip
\section*{P07: ¿Cuál crees que será el nivel de estudios más alto que alcanzarás?}
\subsection*{Options}
\begin{itemize}[leftmargin=*]
  \item 1: No terminaré la primaria
  \item 2: Terminaré la secundaria
  \item 3: Terminaré una carrera técnica
  \item 4: Terminaré una carrera universitaria
  \item 5: Terminaré una maestría o doctorado
\end{itemize}
\bigskip
\section*{P08: En una semana habitual, ¿cuántas horas les dedicas a la lectura en tu TIEMPO LIBRE?}
\subsection*{Options}
\begin{itemize}[leftmargin=*]
  \item 1: No leo en mi tiempo libre
  \item 2: Menos de 2 horas a la semana
  \item 3: Entre 2 y 4 horas a la semana
  \item 4: Entre 5 y 7 horas a la semana
  \item 5: Entre 8 y 10 horas a la semana
  \item 6: Más de 10 horas a la semana
\end{itemize}
\bigskip
\section*{P09: ¿Qué tan de acuerdo estás con los siguientes enunciados sobre las razones por las que lees en tu TIEMPO LIBRE?}
\subsection*{Sub-items}
\begin{itemize}[leftmargin=*]
  \item p09\_01: Leo porque lo disfruto mucho
  \item p09\_02: Leo porque es divertido
  \item p09\_03: Leo porque me gusta mucho
  \item p09\_04: Leo porque creo que es interesante
  \item p09\_05: Leo porque creo que la lectura es útil
  \item p09\_06: Leo porque es importante para mí
  \item p09\_07: Leo porque no quiero decepcionar a los demás
  \item p09\_08: Leo porque me sentiré culpable si no lo hago
  \item p09\_09: Leo porque eso es lo que los demás esperan que haga
  \item p09\_10: Leo porque tengo que demostrarme a mí mismo que puedo sacar buenas notas en Comunicación
  \item p09\_11: Leo porque me castigarán si no leo
  \item p09\_12: Leo porque lo veo necesario para no sentirme mal conmigo mismo
\end{itemize}
\subsection*{Options}
\begin{itemize}[leftmargin=*]
  \item 1: Totalmente en desacuerdo
  \item 2: En desacuerdo
  \item 3: De acuerdo
  \item 4: Totalmente de acuerdo
\end{itemize}
\bigskip
\section*{P10: Cuando estudias un tema para aprender o para dar un examen, ¿qué tan seguido realizas las siguientes acciones?}
\subsection*{Sub-items}
\begin{itemize}[leftmargin=*]
  \item p10\_01: Resalto lo que me parece más importante
  \item p10\_02: Intento memorizar el contenido palabra por palabra
  \item p10\_03: Repito lo que he leído varias veces
  \item p10\_04: Reescribo mis apuntes del libro a mi cuaderno
  \item p10\_05: Intento comprender lo que aprendo
  \item p10\_06: Se me ocurren ejemplos relacionados con lo que estoy aprendiendo
  \item p10\_07: Intento relacionar lo que estoy aprendiendo con cosas que ya sé
  \item p10\_08: Asocio lo que estudio con imágenes o situaciones que creo en mi mente
  \item p10\_09: Intento explicar los conceptos con mis propias palabras
  \item p10\_10: Intento deducir el significado de palabras o frases desconocidas utilizando el propio texto
  \item p10\_11: Después de leer, me explico a mí mismo lo que he leído
  \item p10\_12: Escribo resúmenes
  \item p10\_13: Elaboro mapas conceptuales o diagramas
  \item p10\_14: Explico de qué trata el texto a otros compañeros o personas
\end{itemize}
\subsection*{Options}
\begin{itemize}[leftmargin=*]
  \item 1: Nunca o casi nunca
  \item 2: Algunas veces
  \item 3: Muchas veces
  \item 4: Siempre o casi siempre
\end{itemize}
\bigskip
\section*{P11: ¿Qué tan de acuerdo estás con los siguientes enunciados sobre tu clase de COMUNICACIÓN?}
\subsection*{Sub-items}
\begin{itemize}[leftmargin=*]
  \item p11\_01: Me gusta lo que leo en la clase
  \item p11\_02: Los textos que nos dan en clase son muy difíciles de comprender
  \item p11\_03: En la clase, me dan textos interesantes para leer
  \item p11\_04: Disfruto aprender cosas nuevas en la clase
  \item p11\_05: Los textos que nos dan en clase son muy largos
  \item p11\_06: En general, me interesa lo que leo en la clase
  \item p11\_07: Los textos que nos dan en clase tienen palabras muy difíciles
  \item p11\_08: En las clases, me dan textos con información nueva
  \item p11\_09: Los textos que leemos son para estudiantes de nuestra edad
  \item p11\_10: Los textos que nos dan en clase son muy aburridos
\end{itemize}
\subsection*{Options}
\begin{itemize}[leftmargin=*]
  \item 1: Totalmente en desacuerdo
  \item 2: En desacuerdo
  \item 3: De acuerdo
  \item 4: Totalmente de acuerdo
\end{itemize}
\bigskip
\section*{P12: ¿Qué tan seguido el profesor que enseña COMUNICACIÓN realiza las siguientes acciones?}
\subsection*{Sub-items}
\begin{itemize}[leftmargin=*]
  \item p12\_01: Nos pide dar nuestra opinión personal sobre lo que leemos
  \item p12\_02: Nos pide explicar por qué nos gusta o disgusta un pasaje del texto
  \item p12\_03: Nos pide que pensemos o que nos preguntemos sobre cómo se relacionan los textos que leemos con nuestra vida diaria
  \item p12\_04: Nos pide que expliquemos nuestras respuestas con detalle
  \item p12\_05: Nos presenta textos que incluyen esquemas o mapas conceptuales
  \item p12\_06: Nos pide explicar el aporte de los gráficos o dibujos del texto
  \item p12\_07: Nos presenta diferentes textos que tratan del mismo tema
  \item p12\_08: Nos hace pensar si es que podemos confiar en lo que dice el autor en el texto
  \item p12\_09: Nos pide que compartamos en grupo nuestras opiniones sobre los textos
\end{itemize}
\subsection*{Options}
\begin{itemize}[leftmargin=*]
  \item 1: Nunca o casi nunca
  \item 2: En algunas clases
  \item 3: En la mayoría de las clases
  \item 4: En todas o en casi todas las clases
\end{itemize}
\bigskip
\section*{P13: ¿Qué tan seguido el profesor que enseña MATEMÁTICA realiza las siguientes acciones?}
\subsection*{Sub-items}
\begin{itemize}[leftmargin=*]
  \item p13\_01: Nos deja tareas que requieren pensar más y no solo aplicar fórmulas o resolver ecuaciones
  \item p13\_02: Nos pide explicar cómo hemos resuelto un problema o ejercicio
  \item p13\_03: Nos pide que pensemos o nos preguntemos sobre cómo se relacionan los temas nuevos con los anteriores
  \item p13\_04: Nos pide que expliquemos nuestras respuestas con detalle
  \item p13\_05: Nos enseña a resolver los problemas de muchas maneras
  \item p13\_06: Nos pide que elaboremos nuestros propios problemas de matemática
  \item p13\_07: Nos motiva a que encontremos nuestra propia forma de resolver los problemas o ejercicios de matemática
  \item p13\_08: Nos motiva a relacionar lo que aprendemos con situaciones de la vida real
\end{itemize}
\subsection*{Options}
\begin{itemize}[leftmargin=*]
  \item 1: Nunca o casi nunca
  \item 2: En algunas clases
  \item 3: En la mayoría de las clases
  \item 4: En todas o en casi todas las clases
\end{itemize}
\bigskip
\section*{P14: ¿Qué tan seguido ocurren las siguientes situaciones en las clases de MATEMÁTICA?}
\subsection*{Sub-items}
\begin{itemize}[leftmargin=*]
  \item p14\_01: Durante la clase hay desorden en el salón
  \item p14\_02: El profesor tiene que esperar mucho tiempo para que los estudiantes se callen
  \item p14\_03: Algunos estudiantes hacen tanto desorden que se hace difícil aprender
  \item p14\_04: Cuando el profesor explica algo en clase, es capaz de hacer que todos le prestemos atención
  \item p14\_05: El profesor mantiene el orden de las clases a pesar de todas las preguntas de los estudiantes
  \item p14\_06: Los estudiantes participan de manera ordenada en las clases
  \item p14\_07: Cuando un estudiante participa en clase, otros interrumpen
  \item p14\_08: Los estudiantes se distraen en la clase
\end{itemize}
\subsection*{Options}
\begin{itemize}[leftmargin=*]
  \item 1: Nunca o casi nunca
  \item 2: En algunas clases
  \item 3: En la mayoría de las clases
  \item 4: En todas o en casi todas las clases
\end{itemize}
\bigskip
\section*{P15: ¿Qué tan de acuerdo estás con los siguientes enunciados?}
\subsection*{Sub-items}
\begin{itemize}[leftmargin=*]
  \item p15\_01: La matemática me da miedo
  \item p15\_02: En general, me preocupa resolver ejercicios de matemática
  \item p15\_03: Me pongo muy nervioso o nerviosa en un examen de matemática
  \item p15\_04: Normalmente, la matemática me pone muy nervioso o nerviosa
  \item p15\_05: Me siento mal cuando pienso en resolver problemas de matemática
  \item p15\_06: Cuando hago problemas de matemática se me pone la mente en blanco y no puedo pensar claramente
  \item p15\_07: Me dan miedo los exámenes de matemática
  \item p15\_08: Las tareas de matemática me hacen sentir nervioso o nerviosa
\end{itemize}
\subsection*{Options}
\begin{itemize}[leftmargin=*]
  \item 1: Totalmente en desacuerdo
  \item 2: En desacuerdo
  \item 3: De acuerdo
  \item 4: Totalmente de acuerdo
\end{itemize}
\bigskip
\section*{P16: ¿Qué tan de acuerdo estás con las siguientes afirmaciones relacionadas con la matemática?}
\subsection*{Sub-items}
\begin{itemize}[leftmargin=*]
  \item p16\_01: Me gusta la matemática
  \item p16\_02: Me siento bien cuando tengo que hacer mis tareas de matemática
  \item p16\_03: Me da pena tener que gastar mi tiempo haciendo ejercicios de matemática
  \item p16\_04: Cuando tengo que hacer matemática, preferiría hacer otras cosas
  \item p16\_05: La matemática es fácil para mí
  \item p16\_06: Me intereso por la clase de matemática
  \item p16\_07: Me interesa que me vaya bien en matemática porque es importante para mí
  \item p16\_08: Si me dieran la opción de elegir, escogería aprender matemática
\end{itemize}
\subsection*{Options}
\begin{itemize}[leftmargin=*]
  \item 1: Totalmente en desacuerdo
  \item 2: En desacuerdo
  \item 3: De acuerdo
  \item 4: Totalmente de acuerdo
\end{itemize}
\bigskip
\section*{P17: ¿Qué tan seguido el profesor o profesora que enseña PERSONAL SOCIAL hace lo siguiente?}
\subsection*{Sub-items}
\begin{itemize}[leftmargin=*]
  \item p17\_01: Nos pide recordar de memoria las fechas y los lugares en los que sucedieron los hechos
  \item p17\_02: Nos pide identificar las diversas causas y consecuencias de un hecho
  \item p17\_03: Nos pide aprender de memoria los hechos tal como los explicó
  \item p17\_04: Nos presenta diversas maneras de explicar un mismo hecho
  \item p17\_05: Nos dice que solo existe una verdadera versión de los hechos
  \item p17\_06: Nos pide utilizar información confiable al defender un punto de vista
  \item p17\_07: Nos dice que los personajes importantes son los que solucionan los problemas de la sociedad
\end{itemize}
\subsection*{Options}
\begin{itemize}[leftmargin=*]
  \item 1: Nunca o casi nunca
  \item 2: En algunas clases
  \item 3: En la mayoría de las clases
  \item 4: En todas o en casi todas las clases
\end{itemize}
\bigskip
\section*{P18: ¿Qué tan de acuerdo estás con las siguientes afirmaciones?}
\subsection*{Sub-items}
\begin{itemize}[leftmargin=*]
  \item p18\_01: Considero que todas las personas de la comunidad influyen sobre los hechos que ocurren en la sociedad
  \item p18\_02: Considero que las decisiones de los personajes importantes son las que determinan lo que ocurre en la sociedad
  \item p18\_03: Para comprender un hecho es necesario conocer las características de la época y del lugar en el que ocurre
  \item p18\_04: Algunas ideas y costumbres del pasado se mantienen hasta la actualidad
  \item p18\_05: Las ideas y costumbres de la actualidad son totalmente distintas a las del pasado
  \item p18\_06: Considero que un hecho puede tener varias causas
  \item p18\_07: Considero que un hecho puede tener varias consecuencias
\end{itemize}
\subsection*{Options}
\begin{itemize}[leftmargin=*]
  \item 1: Totalmente en desacuerdo
  \item 2: En desacuerdo
  \item 3: De acuerdo
  \item 4: Totalmente de acuerdo
\end{itemize}
\bigskip
\section*{P19: ¿Qué tan de acuerdo estás con las siguientes afirmaciones?}
\subsection*{Sub-items}
\begin{itemize}[leftmargin=*]
  \item p19\_01: Considero que para comprender un hecho es necesario utilizar información de diversas fuentes
  \item p19\_02: Considero que cualquier información es confiable, sin importar de donde proviene
  \item p19\_03: Las personas explican los hechos desde su punto de vista
  \item p19\_04: Un mismo hecho de la sociedad puede ser comprendido de distinta manera
  \item p19\_05: Existe una sola versión verdadera de los hechos que ocurren en la sociedad
  \item p19\_06: Para defender un punto de vista sobre algún hecho es necesario usar información confiable
  \item p19\_07: Los problemas sociales se pueden resolver de una sola manera
  \item p19\_08: Para solucionar los problemas de la sociedad se debe hacer lo mismo que los países desarrollados
  \item p19\_09: Todas las personas pueden contribuir a la solución de los problemas
  \item p19\_10: La solución a los problemas de la sociedad depende de los personajes importantes
\end{itemize}
\subsection*{Options}
\begin{itemize}[leftmargin=*]
  \item 1: Totalmente en desacuerdo
  \item 2: En desacuerdo
  \item 3: De acuerdo
  \item 4: Totalmente de acuerdo
\end{itemize}
\bigskip
\section*{P20: Acerca de tu escuela, ¿qué tan de acuerdo estás con las siguientes afirmaciones?}
\subsection*{Sub-items}
\begin{itemize}[leftmargin=*]
  \item p20\_01: Prefiero faltar a la escuela
  \item p20\_02: Mi escuela es un lugar donde me siento solo
  \item p20\_03: Preferiría estudiar en otra escuela
  \item p20\_04: Siento que no pertenezco a esta escuela
  \item p20\_05: Me siento orgulloso de estar en esta escuela
  \item p20\_06: Mi escuela es un lugar donde me siento como un extraño
  \item p20\_07: Me siento seguro en el camino cuando voy de mi casa a la escuela
  \item p20\_08: Me siento seguro en los salones de clase de mi escuela
  \item p20\_09: Me siento seguro en otros lugares de la escuela (por ejemplo, patios, pasillos, baños, etc.)
\end{itemize}
\subsection*{Options}
\begin{itemize}[leftmargin=*]
  \item 1: Totalmente en desacuerdo
  \item 2: En desacuerdo
  \item 3: De acuerdo
  \item 4: Totalmente de acuerdo
\end{itemize}
\bigskip
\section*{P21: ¿Qué tan de acuerdo estas con las siguientes afirmaciones acerca de tus profesores?}
\subsection*{Sub-items}
\begin{itemize}[leftmargin=*]
  \item p21\_01: Los profesores echan la culpa a los estudiantes por las cosas que pasan
  \item p21\_02: Los profesores pueden castigar a los estudiantes sin decirles la razón por la que se les está castigando
  \item p21\_03: Los profesores tienen demasiadas normas que no tienen sentido, pero debo obedecerlas
  \item p21\_04: En general, los profesores de esta escuela no son muy pacientes con los alumnos
  \item p21\_05: Los profesores dejan que los estudiantes les respondan de forma inadecuada o insulten
\end{itemize}
\subsection*{Options}
\begin{itemize}[leftmargin=*]
  \item 1: Totalmente en desacuerdo
  \item 2: En desacuerdo
  \item 3: De acuerdo
  \item 4: Totalmente de acuerdo
\end{itemize}
\bigskip
\section*{P22: ¿Qué tan de acuerdo estás con las siguientes afirmaciones sobre tu participación en las clases?}
\subsection*{Sub-items}
\begin{itemize}[leftmargin=*]
  \item p22\_01: Me parece importante participar en las clases para aprender
  \item p22\_02: Prefiero no participar en las clases para evitar errores y burlas de mis compañeros
  \item p22\_03: Pregunto en las clases cuando no comprendo algo
  \item p22\_04: Expreso mis opiniones en las clases aun cuando estas sean diferentes a las demás
\end{itemize}
\subsection*{Options}
\begin{itemize}[leftmargin=*]
  \item 1: Totalmente en desacuerdo
  \item 2: En desacuerdo
  \item 3: De acuerdo
  \item 4: Totalmente de acuerdo
\end{itemize}
\bigskip
\section*{P23: Durante las clases, ¿qué tan seguido tus profesores hacen lo siguiente?}
\subsection*{Sub-items}
\begin{itemize}[leftmargin=*]
  \item p23\_01: Nos motivan a expresar nuestras ideas y opiniones en clase
  \item p23\_02: Se molestan cuando nos equivocamos al participar en clase
  \item p23\_03: Nos hablan mucho y dejan poco espacio para que participemos
  \item p23\_04: Nos escuchan con interés cuando opinamos en clase
  \item p23\_05: Nos motivan a formular preguntas durante la clase
  \item p23\_06: Nos plantean actividades interesantes y entretenidas que motivan nuestra participación
  \item p23\_07: Nos piden que opinemos al final de la clase para no interrumpirlos
  \item p23\_08: Nos demuestran que prefieren la participación de los estudiantes más destacados
\end{itemize}
\subsection*{Options}
\begin{itemize}[leftmargin=*]
  \item 1: Nunca o casi nunca
  \item 2: En algunas clases
  \item 3: En la mayoría de las clases
  \item 4: En todas o en casi todas las clases
\end{itemize}
\bigskip
\section*{P24: Con respecto a tu experiencia escolar este año, ¿con qué frecuencia te sientes de la siguiente manera?}
\subsection*{Sub-items}
\begin{itemize}[leftmargin=*]
  \item p24\_01: Siento que hay demasiadas tareas para la casa
  \item p24\_02: Siento que decepciono a mis padres cuando mis notas en los exámenes son malas
  \item p24\_03: Siento que decepciono a mi profesor cuando mis notas en los exámenes son bajas
  \item p24\_04: Siento que hay demasiadas tareas o actividades que hacer en la escuela
  \item p24\_05: Me preocupa que mis notas sean malas
  \item p24\_06: Siento que hay demasiados exámenes en la escuela
  \item p24\_07: Pienso que la nota es muy importante para mi futuro e incluso puede determinar toda mi vida
  \item p24\_08: Siento que en mi colegio me dejan mucha tarea para la casa
\end{itemize}
\subsection*{Options}
\begin{itemize}[leftmargin=*]
  \item 1: Nunca o casi nunca
  \item 2: Algunas veces
  \item 3: Muchas veces
  \item 4: Siempre o casi siempre
\end{itemize}
\bigskip
\section*{P25: Durante este año, ¿con qué frecuencia un PROFESOR(A) U OTRO ADULTO de tu colegio ha actuado de la siguiente manera? Piensa en situaciones que incluyan contacto cara a cara, mensajes de texto, WhatsApp, redes sociales o por otro medio por internet.}
\subsection*{Sub-items}
\begin{itemize}[leftmargin=*]
  \item p25\_01: Ha amenazado o intimidado a un estudiante
  \item p25\_02: Se ha burlado o ha insultado a un estudiante
  \item p25\_03: Ha castigado a un estudiante golpeándole con su mano u otra parte de su cuerpo (le ha dado un cocacho, un palmazo, una cachetada, una patada, etc.)
  \item p25\_04: Ha castigado a un estudiante golpeándole con un objeto (cuaderno, correa, regla, palo, etc.)
  \item p25\_05: Ha castigado a un estudiante humillándolo o haciéndolo sentir mal (gritándole, ridiculizándolo delante de sus compañeros)
\end{itemize}
\subsection*{Options}
\begin{itemize}[leftmargin=*]
  \item 1: Nunca
  \item 2: Unas cuantas veces al año
  \item 3: Unas cuantas veces al mes
  \item 4: Una o más veces por semana
\end{itemize}
\bigskip
\section*{P26: Durante este año, ¿con qué frecuencia ALGUNOS COMPAÑEROS(AS) han actuado de la siguiente manera con otros(as) estudiantes? Piensa en situaciones que incluyan contacto cara a cara, mensajes de texto, WhatsApp, redes sociales o por otro medio por internet.}
\subsection*{Sub-items}
\begin{itemize}[leftmargin=*]
  \item p26\_01: Han amenazado o intimidado a un estudiante
  \item p26\_02: Han golpeado o herido a un estudiante (jalándole el pelo, dándole cocachos, pateándolo, haciéndolo tropezar, empujándolo, etc.)
  \item p26\_03: Se han burlado o insultado a un estudiante
  \item p26\_04: Han quitado, escondido, robado o roto sus cosas a un estudiante
  \item p26\_05: Han difundido rumores o chismes sobre un estudiante
  \item p26\_06: Han gritado a un estudiante frente a otros
\end{itemize}
\subsection*{Options}
\begin{itemize}[leftmargin=*]
  \item 1: Nunca
  \item 2: Unas cuantas veces al año
  \item 3: Unas cuantas veces al mes
  \item 4: Una o más veces por semana
\end{itemize}
\bigskip
\section*{P27: Cuando las situaciones mencionadas en las preguntas anteriores han sucedido en tu colegio, ¿le contaste y/o solicitaste ayuda a las siguientes personas?}
\subsection*{Sub-items}
\begin{itemize}[leftmargin=*]
  \item p27\_01: Mi tutor(a)
  \item p27\_02: Otro profesor(a)
  \item p27\_03: Al director(a)
  \item p27\_04: Al auxiliar
  \item p27\_05: Al psicólogo(a) de la escuela
  \item p27\_06: Al personal administrativo, de limpieza o seguridad
  \item p27\_07: A un familiar
  \item p27\_08: A un amigo(a)
\end{itemize}
\subsection*{Options}
\begin{itemize}[leftmargin=*]
  \item 1: No
  \item 2: Sí
\end{itemize}
\bigskip
\end{document}
